\documentclass{agora_flyer_a5}

\begin{document}

\AgoraFlyerTitlePage
\sloppy

\Headline{Demokratie beginnt im Kopf}

\begin{Intro}
Jeder Mensch ist anfällig für Manipulation, was sich schon
anhand der allgegenwärtigen Werbung zeigt. Wir sind tagtäglich
von Informationen umgeben, die gezielt gestaltet wurden, um
unsere Aufmerksamkeit zu lenken und uns zu bestimmten
Handlungen zu bewegen. Im Bereich der Konsum-Werbung ist
dies noch relativ harmlos, doch es wird gefährlich, wenn
politische Beeinflussung und Kriegspropaganda an uns
herangetragen werden.
\end{Intro}

\ 

Da in einer Demokratie alle Staatsgewalt vom Volk ausgeht,
bestimmt die öffentliche Meinung über den politischen
Entscheidungsprozess. Es ist deshalb die Pflicht eines jeden
Bürgers, sich kritisch mit den Medien auseinanderzusetzen. Große
Medienkonzerne und öffentlich-rechtliche Anstalten mit
parteinahen Aufsichtsräten bestimmen, welche Themen auf der
Tagesordnung stehen und wie sie zu betrachten sind. Wir als
einfache Bürger haben keine Möglichkeit, auf die Berichterstattung
dieser großen Akteure Einfluss zu nehmen. Während die privaten
Medien fundamental undemokratisch und nur ihrem Konzern-
besitzer verpflichtet sind, lassen auch die öffentlich-rechtlichen ein
pluralistisches Meinungsspektrum in weiten Teilen vermissen. Zu
zahlreichen wichtigen Themen wie der Euro-Rettung, den
Russland-Sanktionen, der Massenmigration und dem Corona-
Lockdown findet man in allen großen Zeitungen und
Fernsehsendern nur eine einheitliche Meinung, welche allzu oft mit
der Regierungsposition identisch ist.

Aufgrund dieser Missstände möchten wir Sie mit folgenden fünf
Punkten zu unabhängigem und kritischem Denken anregen:

\begin{enumerate}
    \item Informieren Sie sich aus zahlreichen Quellen mit verschiedenen
          Positionen. Eine Liste unabhängiger Medien ist hier beigefügt.
    \item Bleiben Sie bei jeder Veröffentlichung, der Sie Ihre
          Aufmerksamkeit schenken, kritisch – auch bei dieser hier – und
          fragen Sie nach den dahinterliegenden Motiven. Betrachten Sie
          bei diffamierender Berichterstattung immer beide Seiten der
          Debatte.
    \item Achten Sie auf die Emotion, welche Sie beim Medienkonsum
          erhalten. Propaganda ist das gezielte Erzeugen von emotionaler
          Zu- und Abneigung.
    \item Lesen Sie Bücher. Wirklich tiefgreifendes Wissen muss unter
          dem Einsatz von Zeit erarbeitet werden. Eine Meinung ohne
          fundiertes Wissen ist wertlos.
    \item Werden Sie aktiv. Die Beteiligung der Bürger ist essenziell für
          das Funktionieren einer Demokratie. Wer nichts tut, um seine
          Ansichten in die Öffentlichkeit zu tragen, dessen Interessen
          werden auch nicht gehört werden.
\end{enumerate}

Mit diesem Flugblatt wollen wir zum demokratischen Diskurs
beitragen. Jeder kann es sich auf folgender Internetseite
herunterladen und es dann selbst verteilen:

\begin{center}
    \large \bfseries www.agora-initiative.de
\end{center}

Nur wer aktiv wird, kann etwas verändern!

Auf der Internetseite stellen wir auch unser „Handbuch der
öffentlichen Meinung“ zum Herunterladen bereit, in dem wir die
psychologischen Techniken der Meinungsmanipulation und deren
detaillierten Hintergründe erläutern. Auch dieses Flugblatt
verwendet Manipulation. Wenn man die Techniken der
Manipulation wie das Framing, das Labeling und die
Fragmentierung nicht kennt, dann sind sie unsichtbar, obwohl sie
tagtäglich vor den eigenen Augen verwendet werden.

Wir stellen hier eine Bandbreite progressiv-linker bis patriotisch-
konservativer Medien vor, die allesamt vom Mainstream-Konsens
abweichende Informationen vermitteln.

Wir ermutigen Sie ausdrücklich dazu, alle hier dargestellten
Quellen eigenständig zu studieren. Eine fundierte Anschauung
kann nur durch eigene Arbeit erreicht werden und nicht dadurch,
dass man sie vorgesetzt bekommt.

Internet-Medien:

\begin{tabular}{ll}
    KenFM:                & \WebLink{www.kenfm.de} \\
    Die Junge Freiheit:   & \WebLink{www.jungefreiheit.de} \\
    Die NachDenkSeiten:   & \WebLink{www.nachdenkseiten.de} \\
    Free21:               & \WebLink{www.free21.org} \\
    Das Compact-Magazin:  & \WebLink{www.compact-online.de} \\
    Tichys Einblick:      & \WebLink{www.tichyseinblick.de} \\
    Eingeschenkt-TV:      & \WebLink{www.eingeschenkt.tv} \\
    Die Sezession:        & \WebLink{www.sezession.de} \\
    Das Rubikon-Magazin:  & \WebLink{www.rubikon.news} \\
\end{tabular}

Buchempfehlungen:

\begin{tabular}{l}
    Uwe Krüger -- Mainstream \\
    Udo Ulfkotte -- Gekaufte Journalisten \\
    Rainer Mausfeld -- Warum schweigen die Lämmer? \\
    Caspar von Schrenck-Notzing -- Charakterwäsche \\
    Daniele Ganser -- Illegale Kriege \\
    Thorsten Schulte -- Kontrollverlust \\
\end{tabular}

\end{document}
